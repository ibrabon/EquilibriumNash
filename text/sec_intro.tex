\sectioncentered*{Вступление}
\addcontentsline{toc}{section}{Вступление}

Актуальность макроэкономических моделей, рассмотренных в работе, обусловлена
тем, что они представляют собой попытку объединить два математических подхода к
моделированию реальных процессов управления и принятия решений. 

Первый --- теоретико-игровой --- подход позволяет рассматривать ситуации
конфликта интересов. Даже предельно упрощенные модели в виде биматричных игр
являются основой для интересных выводов прикладного характера.

Второй подход заключается в рассмотрении конфликта как явления протяженного во
времени, динамического процесса принятия решений противоборствующими сторонами.
Вместе с классическими повторяющимися играми в последнее время активно
изучаются игры на временных шкалах, т. е. игры, в которых время для одного или
всех игроков устроено сложнее, чем просто множество $\mathbb{N}_0$.

В данной дипломной работе, отталкиваясь от известных результатов по изучению
модели Барро---Гордона на временных шкалах, предложена и изучена модель
взаимодействия профсоюза и фирмы--монополиста.  Разработанное программное
обеспечение позволило путем имитационного моделирования изучить свойства
предложенной модели и сформулировать определенные выводы о способах
формирования долгосрочного равновесия.

Структура работы следующая: в разделе 1 кратко излагаются элементы теории игр и
анализа на временных шкалах. Раздел 2 посвящен описанию теоретического базиса
обоих рассмотренных моделей.  В разделе 3 излагаются результаты компьютерных
экспериментов с изученными моделями, далее формулируются выводы.  Завершает
текст работы список использованной литературы и приложение, в котором приведен
прокомментированный исходных код приложения.
