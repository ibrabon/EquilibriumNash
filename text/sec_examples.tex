\section{Компьютерная имитация макроэкономических моделей} 

\subsection{Анализ устойчивости стратегий в модели Барро-Гордона}
Положим, что матрица выигрышей двух игроков соответствует~\ref{table:sec:ot:real}.\\
\begin{table}[h]
	\centering
	\begin{tabular}{|l|l|l|l|}
		\hline
		\multicolumn{2}{|l|}{\multirow{2}{*}{}} & \multicolumn{2}{l|}{Общественность} \\ \cline{3-4} 
		\multicolumn{2}{|l|}{}                  & $L$            & $H$            \\ \hline
		\multirow{2}{*}{Правительство}     & $L$     & $0,0$          & $-1,-1$          \\ \cline{2-4} 
		& $H$     & $\frac{1}{2},-1$          & $-\frac{1}{2},0$          \\ \hline
	\end{tabular}
	\caption{}	
	\label{table:sec:ot:real1}
\end{table}\\
$r^g= 7 $ - количество ходов совершаемое  правительством за игру\\
$r^p= 4 $ - количество ходов совершаемое  обществом за игру\\
Проверим выполнение условия теоремы~\ref{eq:sec:tech:theoremSystem} для матрицы выигрышей~\ref{table:sec:ot:real1}. 

$$
r^g> \bar{r^g}(R) = \left\{ 
\begin{aligned} 
&2r^p= 2r^p, &&\text{если } R=0
\\
&2(1+R)r^p= 2(1+R)r^p, &&\text{если } 	R\in\left(0; \frac{1}{2}\right)
\\
&2Rr^p= 2Rr^p, &&\text{если } 	R\in\left( \frac{1}{2};1\right)
\end{aligned}
\right.		
$$

$$
r^g> \bar{r^g}(R) = 0$$
$$
4 > 0
$$
Следовательно $\frac{r^g}{r^p} \in \left(\frac{3}{2};2\right)$ выбраны правильно и все совершенные равновесия по под-играм должны быть совершенными равновесиями по под-играм Рамсея. \\
 
\begin{itemize}
\item
Рассмотрим случай, когда правительство скорее склонно ввести высокий уровень инфляции, а общественность предполагая, что правительство пойдет на этот шаг с высокой долей вероятности поднимет зарплаты:\\
$q^g =[ 0.5; 0.9; 0.7; 0.5; 0.9; 0.73; 0.8 ]$ - соответствующая скалярная функция \\
$q^p=[ 0.8; 0.9; 0.8; 1 ] $ - соответствующая скалярная функция \\
С помощью программы проведем эмуляцию 100 игр с такими исходными данными.

\begin{figure}[h]
	\centering
		\includegraphics[width=0.5\linewidth]{Public1.png}
			
	\caption{Гистограмма выигрышей, синий - правительство, зеленый - общественность}
	\label{fig:stat1}
\end{figure}

\begin{table}[h]
	\centering
	\begin{tabular}{|l|l|l|}
		\hline
		& Правительство & \multicolumn{1}{c|}{Общественность} \\ \hline
		Среднее                                                           & -14.26       & -9.46                         \\ \hline
		\begin{tabular}[c]{@{}l@{}}Стандартное \\ отклонение\end{tabular} & 5.34       & 4.66                         \\ \hline
		Ассиметрия                                                        & 1.10          & -0.041                          \\ \hline
		Эксцесс                                                           & 1.37        &  -0.37                         \\ \hline
	\end{tabular}
\end{table}

Согласно полученным результатам легко видеть, что подобный набор стратегий невыгоден обеим сторонам, к тому же правительству из-за более "мягкого" подхода "более невыгодно".\\


\item Рассмотрим случай, когда правительство на последнем шаге может отступить от оптимальной для обоих игроков стратегии $(L,L)$.
\\
Для этого положим скалярные функции:\\
$q^g =[ 0; 0; 0; 0; 0; 0; 0.3 ]$ \\
$q^p=[ 0; 0; 0; 0 ] $ \\


\begin{figure}[h]
	

		\includegraphics[width=0.5\linewidth]{Public.png}
	\centering	
	\caption{Гистограмма выигрышей, синий - правительство, зеленый - общественность}
	\label{fig:stat}
\end{figure}

\begin{table}[h]
	\centering
	\begin{tabular}{|l|l|l|}
		\hline
		& Правительство & \multicolumn{1}{c|}{Общественность} \\ \hline
		Среднее                                                           & 0.39        & -0.78                         \\ \hline
		\begin{tabular}[c]{@{}l@{}}Стандартное \\ отклонение\end{tabular} & 0.78         & 1.56                          \\ \hline
		Ассиметрия                                                        & 1.6          & -1.11                          \\ \hline
		Эксцесс                                                           & 0.58        & 0.58                        \\ \hline
	\end{tabular}
\end{table}

Данная стратегия является выигрышной для правительства, если общественность не ожидает подобного хода под конец действия срока правительства.\\

\item Рассмотрим случай, когда общественность доверяет правительству и почти уверенно, что оно не может поднять уровень инфляции к окончанию срока своего правления.\\
Для этого положим скалярные функции:\\
$q^g =[ 0; 0; 0; 0; 0; 0; 0.3 ]$ \\
$q^p=[ 0; 0; 0; 0.15] $ \\
	
	\begin{figure}[h]
		
\centering
				\includegraphics[width=0.5\linewidth]{Public2.png}
		
		\caption{Гистограмма выигрышей, синий - правительство, зеленый - общественность}
		\label{fig:stat2}
	\end{figure}
	\begin{table}[h]
		\centering
		\begin{tabular}{|l|l|l|}
			\hline
			& Правительство & \multicolumn{1}{c|}{Общественность} \\ \hline
			Среднее                                                           & -0.9       &  -2.18                        \\ \hline
			\begin{tabular}[c]{@{}l@{}}Стандартное \\ отклонение\end{tabular} & 3         & 2.58                         \\ \hline
			Ассиметрия                                                        & -1.16          & -0.68                         \\ \hline
			Эксцесс                                                           &  -0.06       & -0.95                        \\ \hline
		\end{tabular}
	\end{table}
	
	
Легко видеть, что данная стратегия однозначно ухудшает позицию правительства, но так же не выгодна общественности.\\


\item Рассмотрим случай, когда общественность не доверяет правительству и почти уверенно, что оно повысит уровень инфляции к окончанию срока своего правления.\\
	 	Для этого положим скалярные функции:\\
	 $q^g =[ 0; 0; 0; 0; 0; 0; 0.3 ]$ \\
	 $q^p=[ 0; 0; 0; 0.8] $ \\
	 
 	\begin{figure}[h]
 		\centering
 			\includegraphics[width=0.5\linewidth]{Public3.png}
 		\caption{Гистограмма выигрышей, синий - правительство, зеленый - общественность}
 		\label{fig:stat3}
 	\end{figure}
 	
 	\begin{table}[h]
 		\centering
 		\begin{tabular}{|l|l|l|}
 			\hline
 			& Правительство & \multicolumn{1}{c|}{Общественность} \\ \hline
 			Среднее                                                           & -4.8      &  -4.62                         \\ \hline
 			\begin{tabular}[c]{@{}l@{}}Стандартное \\ отклонение\end{tabular} &  2.99        &  2.65                          \\ \hline
 			Ассиметрия                                                        & 1.18          &0.54                          \\ \hline
 			Эксцесс                                                           & -0.12       & -1.15                       \\ \hline
 		\end{tabular}
 	\end{table}	

 
 Данная стратегия является равносильно невыгодна для обоих игроков.\\
 
В связи с полученными результатами можно сделать вывод, что если правительство имеет больше шагов на временных шкалах, то ей имеет смысл отклонится от оптимальной по Парето стратегии $L,L$ и повысить уровень инфляции под конец срока своего правления, так как даже если общественность будет ожидать такого хода, то с целью минимизировать свои потери не будет повышать индексирование заработной платы заранее. Естественно если правительство захочет заручиться поддержкой общественности на следующих выборах, то такой ход с её стороны будет сродни "политическому самоубийству".

\end{itemize}
\subsection{Анализ устойчивости стратегий в модели профсоюз---монополист}
Положим матрицу выигрышей в соответствии с~\ref{table:firm} следующим образом:
\begin{table}[h]
	
	\centering
	\begin{tabular}{|l|l|l|l|}
		\hline
		\multicolumn{2}{|l|}{\multirow{2}{*}{}} & \multicolumn{2}{l|}{Профсоюз} \\ \cline{3-4} 
		\multicolumn{2}{|l|}{}                  & $L$            & $H$            \\ \hline
		\multirow{2}{*}{Фирма}     & $L$     & $3, 1$          & $2, 3.9$          \\ \cline{2-4} 
		& $H$     & $7, 4$          & $-3, 7$          \\ \hline
	\end{tabular}
	
\end{table}\\
Для нахождения равновесия по Нэшу посчитаем следующее: 
\begin{itemize}
\item Фирма
	$$ L:  3\alpha + 2(1-\alpha)=\alpha + 2$$
	$$ H: 7\alpha - 3(1-\alpha)=10\alpha - 3$$
	$$10\alpha - 3 = \alpha+2 $$:
	$$\alpha = \frac{5}{9} $$
\item Профсоюз	
	 $$L: \beta + 4(1-\beta)=-3\beta + 4$$
	 $$H: 3.9\beta + 7(1-\beta)=-3.1\beta +7$$
	$$0.1\beta  = 3 $$
	$$\beta = 30 $$
	
	
\end{itemize}

Следовательно равновесием по Нэшу будет стратегия $L,H$.
\newpage
В каждом столбце матрицы фирмы найдем максимальный элемент. 
Затем в каждой строке матрицы профсоюза выберем наибольший элемент.
Платежная матрица фирмы:\\
\begin{table}[h]
	\centering
	\begin{tabular}{|l|l|}
		\hline
		3 1 & \textbf{2 3.9}  \\ \hline
		\textbf{7} 4 & -3 \textbf{7} \\ \hline
	\end{tabular}
\end{table}\\
Оптимальной по Парето будет стратегия $(H,L)$.\\
$r^f= 4 $ - количество ходов совершаемое фирмой за игру\\
$r^p= 3 $ - количество ходов совершаемое профсоюзом за игру\\ \\

\begin{itemize}
\item Рассмотрим случай, когда оба игрока стараются следовать равновесной стратегии по Нэшу:\\
$q^f =[ 0.9; 0.9; 0.9; 0.9 ]$ - соответствующая скалярная функция фирмы\\
$q^p=[ 0.1; 0.1; 0.1] $ - соответствующая скалярная функция профсоюза\\

 	\begin{figure}[h]
 		\centering
 		\includegraphics[width=0.5\linewidth]{Public4.png}
 		\caption{Гистограмма выигрышей, синий - фирма, зеленый - профсоюз}
 		\label{fig:stat4}
 	\end{figure}
 	
 	\begin{table}[h]
 		\centering
 		\begin{tabular}{|l|l|l|}
 			\hline
 			& Фирма & \multicolumn{1}{c|}{Профсоюз} \\ \hline
 			Среднее                                                           & 69.02      &  47.84                        \\ \hline
 			\begin{tabular}[c]{@{}l@{}}Стандартное \\ отклонение\end{tabular} & 18.10        &  8.19                          \\ \hline
 			Ассиметрия                                                        & -1.04          &0.017                          \\ \hline
 			Эксцесс                                                           & 0.25        & 0.69                        \\ \hline
 		\end{tabular}
 	\end{table}
 	
 	\newpage
 \item Рассмотрим случай, когда профсоюз пытается увеличивать зарплату, на своём последнем шаге, а фирма для выхода из нерентабельного положения сокращает количество сотрудников.\\
 $q^f =[ 0.9; 0.9; 0.9; 0.1 ]$ - соответствующая скалярная функция фирмы\\
 $q^p=[ 0.1; 0.1; 0.9] $ - соответствующая скалярная функция профсоюза\\
 \begin{figure}[h]
 	\centering
 	\includegraphics[width=0.5\linewidth]{Public5.png}
 	\caption{Гистограмма выигрышей, синий - фирма, зеленый - профсоюз}
 	\label{fig:stat5}
 \end{figure}
 
 \begin{table}[h]
 	\centering
 	\begin{tabular}{|l|l|l|}
 		\hline
 		& Фирма & \multicolumn{1}{c|}{Профсоюз} \\ \hline
 		Среднее                                                           & 50.48      &  51.1                        \\ \hline
 		\begin{tabular}[c]{@{}l@{}}Стандартное \\ отклонение\end{tabular} & 17.18        &  7.17                          \\ \hline
 		Ассиметрия                                                        & -2.02          &0.57                          \\ \hline
 		Эксцесс                                                           & 5.21        & 1.44                        \\ \hline
 	\end{tabular}
 \end{table}
такая стратегия незначительно улучшит позиции профсоюза, но заметно пошатнет позицию фирмы.

\item Рассмотрим случай, когда профсоюз увеличит зарплату во второй раз и вернется к оптимальной стратегии на третий период:\\
 $q^f =[ 0.9; 0.9; 0.1; 0.9 ]$ - соответствующая скалярная функция фирмы\\
 $q^p=[ 0.1; 0.9; 0.1] $ - соответствующая скалярная функция профсоюза\\
 \begin{figure}[h]
 	\centering
 	\includegraphics[width=0.5\linewidth]{Public6.png}
 	\caption{Гистограмма выигрышей, синий - фирма, зеленый - профсоюз}
 	\label{fig:stat6}
 \end{figure}
 
 \begin{table}[h]
 	\centering
 	\begin{tabular}{|l|l|l|}
 		\hline
 		& Фирма & \multicolumn{1}{c|}{Профсоюз} \\ \hline
 		Среднее                                                           & 43.06      &  50.64                        \\ \hline
 		\begin{tabular}[c]{@{}l@{}}Стандартное \\ отклонение\end{tabular} & 14.47        &  7.38                          \\ \hline
 		Ассиметрия                                                        & -1.06          &-0.29                          \\ \hline
 		Эксцесс                                                           & 1.30       &0.09                        \\ \hline
 	\end{tabular}
 \end{table}
 
 Снова таки разовая акция поднятия зарплат даёт профсоюзу незначительный выигрыш, но фирма при этом проигрывает относительно много больше, чем выигрывает профсоюз.\\ 
 
 \item Рассмотрим случай, когда профсоюз поднимает на втором своём ходе зарплату, но не опускает его вплоть до конца игры:\\ 
 
  $q^f =[ 0.9; 0.9; 0.1; 0.2 ]$ - соответствующая скалярная функция фирмы\\
  $q^p=[ 0.1; 0.9; 0.8] $ - соответствующая скалярная функция профсоюза\\
  \begin{figure}[h]
  	\centering
  	\includegraphics[width=0.5\linewidth]{Public7.png}
  	\caption{Гистограмма выигрышей, синий - фирма, зеленый - профсоюз}
  	\label{fig:stat7}
  \end{figure}
  
  \begin{table}[h]
  	\centering
  	\begin{tabular}{|l|l|l|}
  		\hline
  		& Фирма & \multicolumn{1}{c|}{Профсоюз} \\ \hline
  		Среднее                                                           & 30.37     &  51.37                        \\ \hline
  		\begin{tabular}[c]{@{}l@{}}Стандартное \\ отклонение\end{tabular} & 13.64        &  8.99                          \\ \hline
  		Ассиметрия                                                        & -1.36         &-0.53                          \\ \hline
  		Эксцесс                                                           & 2.74       & 0.8                        \\ \hline
  	\end{tabular}
  \end{table}
  
Фирма снова теряет в относительных деньгах, в то время как профсоюз только незначительно увеличивает своё положение. 
Разумно с точки зрения фирмы сделать предложение профсоюзу не менять уровень зарплат с оптимального, а взамен отплачивать разным видом бонусов. Тогда функция полезности изменится для обоих игроков на равную величину. Фирме стоит подобрать эту величину бонусов так, чтобы остаток между её оптимальным выигрышем после вычитания бонусов был чем-то средним между собственными выигрышами, когда профсоюз устанавливает высокий уровень зарплат и низкий.
\end{itemize}