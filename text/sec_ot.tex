
\section{Теоретико-игровая интерпретация модели Барро-Гордона}

В игре присутствуют два игрока: $p$ - власть, $q$ - общественность, чьими инструментами является инфляция $\pi$  и индексирование заработной платы $\omega$, соответственно. Для упрощения модели положим, что оба не могут прогнозировать будущие. Каждый игрок может оперировать своими инструментами следующими стратегиями: низким $L$  и высоким  $H$ увронем поднятия. В общем виде игра может быть суммирована в виде матрицы выигрышей. 
\begin{table}[h]
	\centering
\begin{tabular}{|l|l|l|l|}
	\hline
	\multicolumn{2}{|l|}{\multirow{2}{*}{}} & \multicolumn{2}{l|}{Public} \\ \cline{3-4} 
	\multicolumn{2}{|l|}{}                  & $L$            & $H$            \\ \hline
	\multirow{2}{*}{Government}     & $L$     & $a,q$          & $b,v$          \\ \cline{2-4} 
	& $H$     & $c,x$          & $d,z$          \\ \hline
\end{tabular}
\end{table}

где параметры $a,b,c,d,q,v,x,z$  - выигрыши удовлетворяющие следующим ограничениям.

\begin{equation}
c>a>d>b, q>v, q\ge z>x
\label{eq:sec:ot:restrictions}
\end{equation}

Самый простой способ описания экономики в данном случае через функцию совокупного предложения Лукаса

\begin{equation}
\label{eq:sec:ot:lucas}
y_t - Y = \lambda(\pi_t - \omega_t)+\varepsilon_t
\end{equation}

где  $\lambda>0, y$  – производительность, $Y$ – естественный уровень производительности, а  $\varepsilon$ - макроэкономический шок близкий к нулю. Коэффициенты дисконтирования игроков $\beta_g$ и $\beta_p$, а их функции  полезности  следующие: 

\begin{equation}
\label{eq:sec:ot:govUtil}
u^g_t=-(\pi_t - \tilde{\pi})^2 + \alpha y_t - \beta(y_t-Y)^2
\end{equation}

\begin{equation}
\label{eq:sec:ot:pubUtil}
u^p_t=-(\pi_t - \omega)^2,
\end{equation}
где $\tilde{\pi}$ - оптимальный уровень инфляции, а $\alpha > 0, \beta > 0$ , который описывает относительный вес между целями власти (стабильной инфляции, высокой производительности и стабильной производительности). Ожидания стандартны, общественность беспокоится о верном ожидании уровня инфляции для поддержки уровня зарплат на рынке. 
\\

Так как нас интересует эффективность политики, то сфокусируемся на долгосрочном исходе игры. Для того, чтобы этого добиться однозначно определим экономику положив $\forall t, \varepsilon_t=0$ , что подразумевает, что мы можем положить $\beta=0$ без потери общности. Из этого следует, что инструмент власти $\pi$  представляет собой выбор средней инфляции.
\\

В стандартной пошаговой игре, в которой игроки могут менять свое поведение в каждый период, мы используем~(\ref{eq:sec:ot:lucas}) "--~(\ref{eq:sec:ot:govUtil}) для получения равновесия:

\begin{equation}
\label{eq:sec:ot:equilibrium}
\pi^*_t= \tilde{\pi} + \frac{\alpha\lambda}{2}= \omega^*_t
\end{equation}

это результат смещения инфляции $\pi^*_t > \tilde{\pi}$. Сфокусировав наше внимание на двух уровнях действий (Чо и Мацуи(2006)), которые являются наиболее естественными кандидатами – оптимальный уровень из ~(\ref{eq:sec:ot:govUtil}) и согласованного по времени из ~(\ref{eq:sec:ot:equilibrium}).

\begin{equation}
\label{eq:sec:ot:optimal}
\pi \in \left\{L=\tilde{\pi}, H=\tilde{\pi}+\frac{\alpha\lambda}{2} \right\} \ni \omega^*_t
\end{equation}

Мы можем, учитывая~(\ref{eq:sec:ot:lucas}) "--~(\ref{eq:sec:ot:pubUtil}) и поделив на $\left(\frac{\alpha\lambda}{2}\right)$  без потери общности вывести соответствующие выигрыши, представленные в таблице ниже:

\begin{equation}
	\label{eq:sec:ot:constraint}
	c>a=0 > d > b,c=-d=-\frac{b}{2}, q>v,q\ge z>x
\end{equation}

\begin{equation}
\label{eq:sec:ot:exampleConstraint}
c=1 > a=0 > d=-1 > b=-2, q=z=0 > v=x=-1,
\end{equation}

\begin{table}[h]
	\centering
	\begin{tabular}{|l|l|l|l|}
		\hline
		\multicolumn{2}{|l|}{\multirow{2}{*}{}} & \multicolumn{2}{l|}{Public} \\ \cline{3-4} 
		\multicolumn{2}{|l|}{}                  & $L$            & $H$            \\ \hline
		\multirow{2}{*}{Government}     & $L$     & $0,0$          & $-1,-1$          \\ \cline{2-4} 
		& $H$     & $\frac{1}{2},-1$          & $-\frac{1}{2},0$          \\ \hline
	\end{tabular}
	\caption{}	
	\label{table:sec:ot:real}
\end{table}


И вне зависимости от $\lambda$ и  $\alpha$ справедливы следующие ограничения для данной игры в дополнение к изначальным~(\ref{eq:sec:ot:restrictions}).

Стандартная пошаговая игра имеет уникальное равновесие по Нэшу $(H,H)$,
которое, однако, неэффективно так как является Парето доминированным, то есть существует такой исход игры, который улучшит состояние одного, при этом не пойдет во вред другим игрокам, в данном случае это «не Нэшовский» исход $(L,L)$.  У власти возникает соблазн создать неожиданную инфляцию, чтобы повысить производительность и снизить уровень безработицы. Так как общественность рационально, то будет ожидать высокую инфляцию – оба игрока будут в проигрыше. 

