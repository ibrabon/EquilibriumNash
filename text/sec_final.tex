\sectioncentered*{Выводы}
\addcontentsline{toc}{section}{Выводы}
В данной работе была рассмотрена игровая интерпретация модели Барро-Гордона на временных шкалах. Разработано  программное обеспечение для множественной имитации игры на временных шкалах. Была впервые введена игровая интерпретация модели "профсоюз---монополист" и расширена на временные шкалы. Было показано, что на временных шкалах монополисту всегда стоит договариваться с профсоюзом о выплате в виде каких-то бонусов, на которые может влиять только монополист, и менять свои функции полезности, поскольку в обратном случае профсоюз всегда будет ставить $H$ уровень зарплат, что абсолютно не выгодно монополисту, так как он потеряет больше, чем если бы отдал часть выручки на бонусы.