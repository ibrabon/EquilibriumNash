\section{Программная реализация} 

\subsection{Информация о программе}
В рамках дипломной работы была разработана программа, которая эмулирует биматричную игру на временных шкалах и доказывает эмпирически изложенные в работе результаты. 
\\

Разработка программного продукта велась на языке Python 3.4.
Программа позваляет пользователю задать $r^g$, $r^p$ и соответсвующие скалярные функции поведения игрока в определенный момент времени. Значениями данной функции будет вероятность выбора игроком  стратегии $H$ в данный период времени. Так же пользователь может задать количество запусков игры с входящими данными для подсчета статистики. В качетсве интерфейса входящих данных был выбран txt файл.

\subsection{Пояснения к коду}
 \begin{lstlisting}[style=csharpinlinestyle]
	 util.fromFileToMap(filepath) 
 \end{lstlisting}
Функция парсит входящие данных

 \begin{lstlisting}[style=csharpinlinestyle]
	 nash.calculate_nash(government_payoffs, public_payoffs)
 \end{lstlisting}
 Функция рассчитывает равновесие по Нэшу для стандартной игры.
 
 
 \begin{lstlisting}[style=csharpinlinestyle]
	 nash.time_scales_game(request, government_payoffs, public_payoffs))
 \end{lstlisting}
 Имитация игры на временных шкалах, результатом который есть вектор выигрышей игроков за время игры
 
  \begin{lstlisting}[style=csharpinlinestyle]
	  util.create_all_stats(array, player_name, time)
  \end{lstlisting}
 Рассчитывает статистические параметры, строит гистограму для выигрышей одного игрока.
 
 \subsection{Листинг}
 https://github.com/ibrabon/EquilibriumNash/