\section{Основные определения}

\begin{definition}	$(S,H)$— некооперативная игра $n$ лиц в нормальной форме, где $S$ — набор чистых стратегий, а $H$ — набор выигрышей. Когда каждый игрок $i \in \left\{1,...,n\right\}$  выбирает стратегию $x_i \in S$  в профиле стратегий $x=(x_1,...,x_n)$, игрок $i$  получает выигрыш $H_i(x)$. Профиль стратегий $x^* \in S$   является равновесием по Нэшу, если изменение своей стратегии с $x_i^*$  на $x_i$  не выгодно ни одному игроку $i$, то есть $\forall i : H_i(x^*) \ge H_i(x_i, x_{-i}^*)$.
\end{definition}
	
\begin{definition}
Равновесие по Нэшу называется безупречным по под-играм, если и только если оно является равновесием по Нэшу для каждой под-игры.
\end{definition}

\begin{definition}
Парето-оптимальность в смысле теории игр характеризует такую ситуацию, при которой невозможно улучшить исход игры для одного игрока без его ухудшения для других игроков.
\end{definition}

\begin{definition}
	Инфляция  — повышение общего уровня цен на товары и услуги.
\end{definition}

\begin{definition}
	Индексация заработной платы – повышение заработных плат с целью частичной защиты населения от роста потребительских цен на товары и услуги.
\end{definition}

\begin{definition}
	Любое совершенное равновесие по под-играм (SPNE), в котором оба игрока выбирают стратегию L во всех своих ходах, назовём \textbf{совершенным равновесием Рамси по под-играм (Ramsey SPNE)}
\end{definition}
