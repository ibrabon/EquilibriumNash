\section{Основные понятия и определения}
\subsection{Элементы теории игр}

\begin{definition}
	$(S,H)$ --- не кооперативная игра $n$ лиц в нормальной
	форме, где $S$ --- набор чистых стратегий, а $H$ --- набор выигрышей. Когда
	каждый игрок $i \in \left\{1,\dots,n\right\}$  выбирает стратегию $x_i
	\in S$  в профиле стратегий $x=(x_1,\dots,x_n)$, игрок $i$  получает
	выигрыш $H_i(x)$. Профиль стратегий $x^* \in S$   является равновесием
	по Нэшу, если изменение своей стратегии с $x_i^*$  на $x_i$  не выгодно
	ни одному игроку $i$, то есть $\forall i : H_i(x^*) \geqslant H_i(x_i,
	x_{-i}^*)$.
\end{definition}
	
\begin{definition}
	Равновесие по Нэшу называется безупречным по подыграм, если и только если оно
	является равновесием по Нэшу для каждой подыгры. 
\end{definition}

\begin{definition}
Парето-оптимальность в смысле теории игр характеризует такую ситуацию, при которой невозможно улучшить исход игры для одного игрока без его ухудшения для других игроков.
\end{definition}

\begin{definition}
	Инфляция --- повышение общего уровня цен на товары и услуги.
\end{definition}

\begin{definition}
	Индексация заработной платы --- повышение заработных плат с целью частичной защиты населения от роста потребительских цен на товары и услуги.
\end{definition}

\begin{definition}
	Любое совершенное равновесие по подыграм (SPNE), в котором оба игрока выбирают стратегию L во всех своих ходах, назовём \textbf{совершенным равновесием Рамсея по под-играм (Ramsey SPNE)}~\cite{libich2008macroeconomic}
\end{definition}

\subsection{Элементы анализа на временных шкалах}

Сведения из теории временных шкал приводятся следуя двум основным
источникам \cite{Bohner,BohnerAdv}.

\begin{definition}
	Под временной шкалой понимается непустое замкнутое подмножество множества
	вещественных чисел, она обозначается символом $\mathbb{T}$.
	Свойства временной шкалы определяются тремя функциями:
	\begin{itemize}
		\item[1)] оператор перехода вперед:
		\[
		\sigma(t) = \inf\left\{s \in \mathbb{T}: s > t\right\};
		\]
		\item[2)] оператор перехода назад:
		\[
		\rho(t) = \sup\left\{s \in \mathbb{T}: s < t\right\},
		\]
		(при этом полагается $\inf\varnothing = \sup{\mathbb{T}}$ и $\sup\varnothing = \inf{\mathbb{T}}$);
		\item[3)] функция зернистости
		$$\mu(t) = \sigma(t) - t.$$
	\end{itemize}
\end{definition}
Поведение операторов перехода вперед и назад в конкретной точке временной шкалы
определяет тип этой точки. Соответствующая классификация точек представлена в
таблице~(\ref{tab:pointclass}).
\begin{table}[h]
	\begin{center}
		\begin{tabular}{|l|l|}
			\hline
			$t$ справа рассеянная & $t < \sigma(t)$ \\
			$t$ справа плотная    & $t = \sigma(t)$ \\
			$t$ слева рассеянная  & $ \rho(t) < t $ \\
			$t$ слева плотная     & $ \rho(t) = t $ \\
			$t$ изолированная      & $\rho(t) < t < \sigma(t)$ \\
			$t$ плотная           & $\rho(t) = t = \sigma(t)$ \\
			\hline
		\end{tabular}
	\end{center}
	\caption{Классификация точек временной шкалы}		
	
	\label{tab:pointclass}
\end{table}

\begin{definition}
	 Множество $\mathbb{T}^\kappa$ определим как:
	 \[
	 \mathbb{T}^\kappa =
	 \begin{cases}
	 \mathbb{T}\setminus \left\{M\right\}, & \text{ если }
	 \exists \text{ справа рассеянная точка } M \in \mathbb{T}:\\
	 & M = \sup\mathbb{T}, \sup\mathbb{T}<\infty  \\
	 \mathbb{T} , & \text{ в противном случае}.
	 \end{cases}
	 \]
	 Далее полагаем $\left[a, b\right] =
	 \left\{t \in \mathbb{T} : a \leqslant t \leqslant b\right\}$.
\end{definition}

\begin{definition}
    Пусть $f:\mathbb{T} \rightarrow \mathbb{R}$ и $t \in \mathbb{T}^\kappa$.
    Число $f^{\Delta}(t)$ называется $\Delta$-производной функции $f$ в точке $t$,
    если $\forall \varepsilon > 0$ найдется такая окрестность
    $U$ точки $t$ (то есть, $U = (t - \delta, t + \delta) \cap \mathbb{T}, \delta < 0$),
    что
    \[
    \left|f(\sigma(t)) - f(s) - f^{\Delta}(t)(\sigma(t)-s)\right| \leqslant
        \varepsilon\left|\sigma(t) - s\right| \quad \forall s \in U.
    \]
\end{definition}

\begin{definition}
    Если $f^{\Delta}(t)$ существует $\forall t \in \mathbb{T}^\kappa$,
    то $f:\mathbb{T} \rightarrow \mathbb{R}$ называется $\Delta$-дифферен\-ци\-ру\-емой
    на $\mathbb{T}^\kappa$. Функция $f^{\Delta}(t): \mathbb{T}^\kappa \rightarrow \mathbb{R}$
    называется дельта-производной функции $f$ на $\mathbb{T}^\kappa$.
\end{definition}

Если $f$ дифференцируемая в $t$, то
\[
    f(\sigma(t)) = f(t) + \mu(t)f^{\Delta}(t).
\]

\begin{definition}
    Функция $f:\mathbb{T} \rightarrow \mathbb{R}$ называется регулярной,
    если во всех плотных справа точках временной шкалы $\mathbb{T}$ она имеет
    конечные правосторонние пределы, а во всех слева плотных точках она имеет
    конечные левосторонние пределы.
\end{definition}

\begin{definition}
    Функция $f:\mathbb{T} \rightarrow \mathbb{R}$ называется $rd$-непрерывной,
    если в справа плотных точках она непрерывна, а в слева плотных точках имеет
    конечные левосторонние пределы. Множество таких функций обозначается
    $C_{rd} = C_{rd}(\mathbb{T})$, а множество дифференцируемых функций,
    производная которых $rd$-непрерывна, обозначается как
    $C_{rd}^1 = C_{rd}^1(\mathbb{T})$.
\end{definition}


\begin{definition}
Для любой регулярной функции $f(t)$ существует функция $F$,
дифференцируемая в области $D$ такая, что для всех $t \in D$
выполняется равенство $$F^{\Delta}(t) = f(t).$$
Эта функция называется пред-первообразной для $f(t)$ и
определяется она неоднозначно.
\end{definition}

Неопределенный интеграл на временной шкале имеет вид:
\[
    \int{f(t) \Delta t} = F(t) + C,
\]
где $C$ --- произвольная константа интегрирования, а $F(t)$ --- пред-перво\-образ\-ная для $f(t)$.
Далее, если для всех $t \in \mathbb{T}^\kappa$ выполняется $F^{\Delta}(t) =
f(t)$, где $f:\mathbb{T}\rightarrow\mathbb{R}$ --- $rd$-непрерывная функция, то
$F(t)$ называется первообразной функции $f(t)$. Если $t_0 \in \mathbb{T}$, то
$\displaystyle F(t) = \int\limits_{t_0}^t{f(s) \Delta s}$ для всех $t$.
Определенный $\Delta$-интеграл для любых $r,s \in \mathbb{T}$ определяется как
\[
    \int\limits_r^s{f(t)\Delta t} = F(s) - F(r).
\]
