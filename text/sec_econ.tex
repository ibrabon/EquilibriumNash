\section{Основные понятия и определения}
\subsection{Элементы теории игр}

\begin{definition}	$(S,H)$— не кооперативная игра $n$ лиц в нормальной форме, где $S$ — набор чистых стратегий, а $H$ — набор выигрышей. Когда каждый игрок $i \in \left\{1,...,n\right\}$  выбирает стратегию $x_i \in S$  в профиле стратегий $x=(x_1,...,x_n)$, игрок $i$  получает выигрыш $H_i(x)$. Профиль стратегий $x^* \in S$   является равновесием по Нэшу, если изменение своей стратегии с $x_i^*$  на $x_i$  не выгодно ни одному игроку $i$, то есть $\forall i : H_i(x^*) \geqslant H_i(x_i, x_{-i}^*)$.
\end{definition}
	
\begin{definition}
Равновесие по Нэшу называется безупречным по под-играм, если и только если оно является равновесием по Нэшу для каждой под-игры.
\end{definition}

\begin{definition}
Парето-оптимальность в смысле теории игр характеризует такую ситуацию, при которой невозможно улучшить исход игры для одного игрока без его ухудшения для других игроков.
\end{definition}

\begin{definition}
	Инфляция  — повышение общего уровня цен на товары и услуги.
\end{definition}

\begin{definition}
	Индексация заработной платы – повышение заработных плат с целью частичной защиты населения от роста потребительских цен на товары и услуги.
\end{definition}

\begin{definition}
	Любое совершенное равновесие по под-играм (SPNE), в котором оба игрока выбирают стратегию L во всех своих ходах, назовём \textbf{совершенным равновесием Рамсея по под-играм (Ramsey SPNE)}~\cite{libich2008macroeconomic}
\end{definition}

\subsection{Элементы анализа на временных шкалах}
\begin{definition}
	Под временной шкалой понимается непустое замкнутое подмножество множества
	вещественных чисел, она обозначается символом $\mathbb{T}$.
	Свойства временной шкалы определяются тремя функциями:
	\begin{itemize}
		\item[1)] оператор перехода вперед:
		\[
		\sigma(t) = \inf\left\{s \in \mathbb{T}: s > t\right\};
		\]
		\item[2)] оператор перехода назад:
		\[
		\rho(t) = \sup\left\{s \in \mathbb{T}: s < t\right\},
		\]
		(при этом полагается $\inf\varnothing = \sup{\mathbb{T}}$ и $\sup\varnothing = \inf{\mathbb{T}}$);
		\item[3)] функция зернистости
		$$\mu(t) = \sigma(t) - t.$$
	\end{itemize}
\end{definition}
Поведение операторов перехода вперед и назад в конкретной точке временной шкалы
определяет тип этой точки. Соответствующая классификация точек представлена в
таблице~(\ref{tab:pointclass}).
\begin{table}[h]

	\centering		
		\caption{}	
		\footnotesize Классификация точек временной шкалы\\
		\normalsize
		\begin{tabular}{|l|l|}
			\hline
			$t$ справа рассеянная & $t < \sigma(t)$ \\
			$t$ справа плотная    & $t = \sigma(t)$ \\
			$t$ слева рассеянная  & $ \rho(t) < t $ \\
			$t$ слева плотная     & $ \rho(t) = t $ \\
			$t$ изолированная      & $\rho(t) < t < \sigma(t)$ \\
			$t$ плотная           & $\rho(t) = t = \sigma(t)$ \\
			\hline
		\end{tabular}
	
	\label{tab:pointclass}
\end{table}

\begin{definition}
	 Множество $\mathbb{T}^\kappa$ определим как:
	 \[
	 \mathbb{T}^\kappa =
	 \begin{cases}
	 \mathbb{T}\setminus \left\{M\right\}, & \text{ если }
	 \exists \text{ справа рассеянная точка } M \in \mathbb{T}:\\
	 & M = \sup\mathbb{T}, \sup\mathbb{T}<\infty  \\
	 \mathbb{T} , & \text{ в противном случае}.
	 \end{cases}
	 \]
	 Далее полагаем $\left[a, b\right] =
	 \left\{t \in \mathbb{T} : a \leqslant t \leqslant b\right\}$.
\end{definition}

\begin{definition}
	$\delta$-разбиение отрезка временной шкалы $[t_0, t_f]$
	определим следующим образом: задавшись диаметром разбиения
	$\delta$, положим первую точку разбиения равной $t_0$, а
	дальнейшие точки определим по формуле
	\[
	t_i = \left\{
	\begin{aligned}
	\sup{\left(t_{i-1}, t_{i-1}+\delta\right]}&, \qquad
	\text{если}\quad t_{i-1}+ \delta \in \mathbb{T}^\kappa, \\
	\sigma(t_{i-1}) &, \qquad \text{если}\quad t_{i-1}+\delta \notin \mathbb{T}^\kappa.
	\end{aligned}
	\right.
	\]
\end{definition}