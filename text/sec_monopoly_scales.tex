\section{Модель монополии профсоюза на временных шкалах}

В игре, как и в непрерывной модели присутсвует два игрока: профсоюз $P$ и фирма $F$, чьими рычагами влияния на игру являются $W$ - зарплата рабочего и $E$ - количество нанятых рабочих соответственно.
Каждый игрок выбирает из следующих стратегий: низким $L$ и высоким $H$ уровнем повышения. В общем виде игра может быть задана следующей матрицей выигрышей: 

\begin{table}[h]
	\centering
	\begin{tabular}{|l|l|l|l|}
		\hline
		\multicolumn{2}{|l|}{\multirow{2}{*}{}} & \multicolumn{2}{l|}{Union} \\ \cline{3-4} 
		\multicolumn{2}{|l|}{}                  & $L$            & $H$            \\ \hline
		\multirow{2}{*}{Firm}     & $L$     & $a,q$          & $b,v$          \\ \cline{2-4} 
		& $H$     & $c,x$          & $d,z$          \\ \hline
	\end{tabular}
\end{table}  

Функция полезности профсоюза  $U_t(W,E)=\lambda WE$, где $\lambda \in(0;1)$:
$$\frac{\partial U}{\partial W} > 0; \quad \frac{\partial U}{\partial E}~>~0 \quad U(0,E)=U(W,0)=U(0,0)=0,$$
при этом 
$$U(L,L) < U(L,H) \le U(H, L) < U(H,H) $$

Функция полезности фирмы $\Pi_t(W,E)=cP(\bar{K},E)-WE$:
$$P(\bar{K}, E)=A\bar{K}^\alpha E^\beta$$, где $A$ – коэффициент нейтрального технического прогресса, $\alpha$ и $\beta$ – коэффициенты эластичности валового внутреннего продукта по капитальным и трудовым затратам.\\

$$\Pi(L,L)>\Pi(H,L)$$
$$\Pi(H,H)<\Pi(L,H)$$

Алексей Павлович, не могу понять, как можно вывести единое правило, если всё зависет от разных коэффициентов, ведь соотношение между функциями зависят от того, какое слогаемое из $\Pi_t(W,E)=cP(\bar{K},E)-WE$ растет быстрее.