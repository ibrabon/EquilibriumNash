
\section{Дискретные однородные временные шкалы} 

Рассмотрим игровую интерпритацию модели Барро-Гордона на временных шкалах.

Все допущения выдвинутые в стандартной игре остаются. Расширим их:
\begin{itemize}
\item игра начинается одновременным ходом, 
\item заранее известно незименное количество ходов $r^g \in \mathbb{N}$ и $r^p \in \mathbb{N}$,
\item игра заканчивается через $T$ периодов, где $T$ - наименьшее общее кратное для $r^g$ и $r^p$,
\item игроки рациональны, обладают равноценными знаниями и полной информацией о структуре игры, матрице выигрышей и всех предыдуущих ходах.
 \end{itemize}

Другими словами определяется три временных шкалы: правительства, общественности и самой игры:
\begin{equation}
\label{eq:sec:tech:scales}
T_g = \{0,r^g,2r^g,...,T\}, T_p=\{0,r^p,2r^p,...,T\}, T=T_g\cup T_p 
\end{equation}

Главным преимуществом использования однородных временных шкал является экономическая интерпретация: $r^g$ и $r^p$ представляют собой степень возможности изменений политики относительно инфляции и степень реагирования для внесения изменений в заработные платы.
\\

Ассинхронная игра на временных шкалах будет как правило иметь несколько равновесий по Нэшу, среди которых мы выберем лучшую в зависимости от под-игры.

\begin{theorem}
	Рассмотрим общую несогласованную по времени игру на однородных временных шкалах, для которой выполняются~(\ref{eq:sec:ot:constraint}) и ~(\ref{eq:sec:tech:scales}). Тогда все SNPE игры будут SNPE Рамси, если и только если
	
	\begin{equation}
		\label{eq:sec:tech:theoremSystem}
		r^g> \bar{r^g}(R) = \left\{ 
		\begin{aligned} 
			&\frac{c - d}{a-d}r^p= \frac{a-b}{a-d}r^p, &&\text{если } R=0
			\\
			&\frac{(1+R)(c-d)}{a-d}r^p= \frac{a-b + R(c-d)}{a-d}r^p, &&\text{если } 	R\in(0; \bar{R})
			\\
			&\frac{c-d-(1-R)(a-b)}{a-d}r^p= \frac{(a-b)}{a-d}Rr^p, &&\text{если } 	R\in(\bar{R};1)
		\end{aligned}
		\right.		
	\end{equation}
\end{theorem}
где $\bar{R}=\frac{q-v}{z-x+q-v}$. В несогласованной игре, где справедливо ~(\ref{eq:sec:ot:constraint}),  ~(\ref{eq:sec:tech:theoremSystem}) преобразуется в 

\begin{equation}
	\frac{r^g}{r^p} \in \left(\frac{3}{2}, 2\right)\cup \left(\frac{5}{2}, \infty\right)
\end{equation}
Докозательство: смотреть Либиха и Штелиха ~\cite{libichIncorpo}.