
\section{Дискретные однородные временные шкалы} 

Рассмотрим игровую интерпритацию модели Барро-Гордона на временных шкалах.

Все допущения выдвинутые в стандартной игре остаются. Расширим их:
\begin{itemize}
\item игра начинается одновременным ходом. 
\item заранее известно незименное количество ходов $r^g \in \mathbb{N}$ и $r^p \in \mathbb{N}$.
\item игра заканчивается через $T$ периодов, где $T$ - наименьшее общее кратное для $r^g$ и $r^p$
\item игроки рациональны, обладают равноценными знаниями и полной информацией о структуре игры, матрицей выигрышей, и могут мониторить все предшествующие ходы
 \end{itemize}

Другими словами определяется три временных шкалы: правительства, общественности и самой игры:
\begin{equation}
\label{eq:sec:tech:scales}
T_g = \{0,r^g,2r^g,...,T\}, T_p=\{0,r^p,2r^p,...,T\}, T=T_g\cup T_p 
\end{equation}

Главным преимуществом использования однородных временных шкал это экономическая интерпретация: $r^g$ и $r^p$ представляют собой степень приверженности (???) правительства и степень жесткости поднятия заработных плат общественности (????).