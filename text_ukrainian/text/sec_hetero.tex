
Можно утверждать, что действия игроков могут не всегда быть детерминистическими и/или что частота их ходов различается во времени или подчиняется некоторому случайному процессу. Разнородные временные шкалы позволяют изучать подобные случаи. Для большей эффективности нормализуем горизонт планирования как $T = r^g$. Как и ранее игроки $g$ и $p$ ходят одновременно в первом периоде и во всех $T = r^g$ периодах, в промежутках между которыми общественность так же способна реагировать на ход правительства. Для удобства сравнения оставим все предположения предыдущего параграфа неизменными.

Рассмотрим произвольную временную шкалу $\mathbb{T}$ и произвольную невозрастающую функцию реакции общественности $f : \mathbb{T} \to [0,1]$. В данной работе рассматриваются два представляющих интерес особых случая. В первом, при разнородной (атомистической) общественности, функция реакции может быть интерпретирована как часть общественности, которая уже имела возможность совершить ход. Во втором, при вероятностных ходах общественности, функция реакции может быть интерпретирована как кумулятивная функция распределения её ходов.

Чтобы удостовериться в том, что все SPNE являются SPNE Рамсея, достаточно показать, что в первом ходе оптимальной стратегией $g$ является $L$ вне зависимости от хода общественности в первом периоде. Получим два соответствующих условия:
\begin{equation}
\label{sec:hetero:main1}
ar^g > c \int_0^{r^g} 1 - f(t) \Delta t + d  \int_0^{r^g} f(t) \Delta t ,
\end{equation}
\begin{equation}
\label{sec:hetero:main2}
dr^g > b \int_0^{r^g} 1 - f(t) \Delta t + a  \int_0^{r^g} f(t) \Delta t .
\end{equation}

\begin{theorem}
\label{spneTh}
  Рассмотрим общую несогласованную по времени игру, в которой выполняется~\eqref{eq:sec:ot:constraint} и $f : \mathbb{T} \to [0,1]$ -- невозрастающая функция реакции. Тогда все SPNE игры являются SPNE Рамсея тогда и только тогда, когда выполняется неравенство
\begin{equation}
\label{sec:hetero:main3}
\int_0^{r^g} f(t) \Delta t > \frac{r^g}{2} .
\end{equation}
\end{theorem}

Для лучшей наглядности теоремы~\ref{spneTh} и лучшего понимания приведём несколько следствий и рассмотрим пример. Как уже было указано ранее, мы концентрируемся на двух случаях: разнородной общественности и вероятностных моделях.

\subsection{Разнородная общественность.} Во-первых, предположим, что существует $N$ различных общественных групп (профсоюзов) $p_1,...,p_N$ с соответствующими $r^{p_1},...,r^{p_N}$ и размерами $s_1,...,s_N \in [0,1]$, которые удовлетворяют естественному предположению $\sum_{i=1}{N}s_i = 1$. Чтобы свести наше внимание к первому ходу правительства, предположим что $r^g$ является кратным для $r^{p_i}$ для всех $i = 1,...,N$, то есть
$$ \frac{r^g}{r^{p_i}} \in \mathbb{N}.$$
Для более асинхронных случаев  см.~\cite{libichIncorpo}. Для данного особого случая теорему~\ref{spneTh} можно переписать следующим образом.\\
\textbf{Следствие 1.} \textit{Рассмотрим общую несогласованную по времени игру, в которой выполняется~\eqref{eq:sec:ot:constraint} и общественность состоит из $N$ различных групп. Тогда все SPNE игры являются SPNE Рамсея тогда и только тогда, когда выполняется неравенство
\begin{equation}
\label{sec:hetero:main4}
\sum_{i=1}^N s_i(r^g - r^{p_i}) \geq \frac{r^g}{2} .
\end{equation}
}
%опять же доказательство

\subsection{Вероятностная модель.}
Вернёмся к случаю с унифицированной общественностью, чтобы рассмотреть вероятностные ходы отдельно от влияния эффектов разнородной общественности. В зависимости от результатов переговоров общественность будет реагировать на первый ход правительства в какой-то момент времени из интервала $[a,b], 0< a < b < r^g$, с равномерно распределённым вероятностным законом. Теорема~\ref{spneTh} принимает вид:\\
\textbf{Следствие 2.} \textit{Рассмотрим общую несогласованную по времени игру, в которой выполняется~\eqref{eq:sec:ot:constraint} и общественность делает второй ход в какой-то момент времени из интервала $[a,b]$ с равномерно распределённым вероятностным законом. Тогда все SPNE игры являются SPNE Рамсея тогда и только тогда, когда выполняется неравенство
\begin{equation}
\label{sec:hetero:main5}
\frac{b^2 - a^2}{2} \geq b - \frac{r^g}{2} .
\end{equation}
}
Очевидно, что можно рассматривать произвольные комбинации описанных выше подходов, то есть разнородных игроков с вероятностными ходами, вероятность которых может быть описана некоторыми функциями распределения. Приведём пример.\\
\textbf{Пример 1.} Рассмотрим экономику с двумя профсоюзами, каждый их которых состоит из $s_1$ и $s_2$ рабочих соответственно, так что выполняется $s_1 + s_2 = 1$. Примем в качестве одного периода квартал (раз в квартал выпускаются сводки по макроэкономике). Предположим, что $r^g = 5$ и что время реакции профсоюзов составляют как минимум  1 и 3 квартала, но как максимум -- 1,5 и 3,5 квартала соответственно. Для простоты предположим, что решение, принятое за полтора квартала подчиняется равномерному закону распределения. Тогда получим:
$$ \mathbb{T} :={0} \cup [1, 1.5] \cup {r^g} ,$$
$$ \mathbb{T} :={0} \cup [3, 3.5] \cup {r^g} .$$
Функция реакции тогда имеет вид $f : \mathbb{T}_1\cup\mathbb{T}_2\to[0,1]$:
$$ f(x)\left\{  
\begin{array}{l c l}
0 & if & x = 0,\\
2s_1(x - 1) & if & x \in [1,1.5],\\
s_1 + 2s_2(x - 3) & if & x \in [3,3.5],\\
1 & if & x = 5.
\end{array} 
\right.$$
Интегрируя по $\mathbb{T}_1\cup\mathbb{T}_2$ получим
$$ \int_0^5 f(x) \Delta x = \frac{15}{4}s_1 + \frac{7}{4}s_2. $$
Предположение~\eqref{sec:hetero:main3} теоремы~\ref{spneTh} удовлетворяется (и следовательно единственное равновесие является равновесием Рамсея) тогда и только тогда, когда выполняется
$$ s_1 \geq \frac{3}{8}, \text{или, что эквивалентно,} s_2 \leq \frac{5}{8}. $$
Интуитивно понятно, что профсоюз, принимающий решения быстрее, должен быть достаточно велик для того, чтобы общая реакция общественности была достаточно быстрой, чтобы препятствовать увеличению инфляции правительством.