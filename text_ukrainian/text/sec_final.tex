\sectioncentered*{Висновки}
\addcontentsline{toc}{section}{Висновки }

У даній роботі було розглянуто ігрову інтерпретацію моделі Барро-Гордона на часових шкалах. Нами було вперше введено ігрову інтерпретацію моделі <<профсоюз -- монополіст>>. Ця модель також була розглянута і на часових шкалах. Було розроблено програмне забезпечення для імітаційного моделювання розглянутих макроекономічних ігр.

В результаті проведенного дослідження було встановлено наступне: на часових шкалах монополістові варто домовлятись з профсоюзом про додаткові виплати (наприклад, у вигляді бонусів) при високому рівні найму замість встановлення низького рівня найму; у протилежному разі профсоюз завжди буде встановлювати високий рівень заробітних плат, що не вигідно монополістові, так як він втрачатимє більше, ніж на бонуси.
