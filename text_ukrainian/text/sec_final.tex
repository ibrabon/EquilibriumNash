\sectioncentered*{Выводы}
\addcontentsline{toc}{section}{Выводы}

В данной работе была рассмотрена игровая интерпретация модели Барро-Гордона на
временных шкалах.  Нами впервые введена игровая интерпретация модели <<профсоюз
--- монополист>>. Эта модель была рассмотрена и на временные шкалах.
Разработано программное обеспечение для имитационного моделирования
рассмотренных макроэкономических игр. 

В результате проведенной работы было показано, что на временных шкалах
монополисту стоит договариваться с профсоюзом о дополнительных выплатах
(например, в виде бонусов) при низком уровне найма вместо установления высокого
уровня найма.  В противном случае профсоюз всегда будет устанавливать высокий
уровень зарплат, что невыгодно монополисту, так как он потеряет больше, чем
если бы отдал часть выручки в виде бонусов.
