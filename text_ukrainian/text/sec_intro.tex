\sectioncentered*{Вступ}
\addcontentsline{toc}{section}{Вступ}

Актуальність макроекономічних моделей, розглянутих у роботі, обумовлена
тим, що вони являють собою спробу об'єднати два математичних підходу до
моделювання реальних процесів управління і прийняття рішень. 

Перший --- теоретико-ігровий --- підхід дозволяє розглядати ситуації
конфлікту інтересів. Навіть гранично спрощені моделі у вигляді біматричних ігор
є основою для цікавих висновків прикладного характеру.

Другий підхід полягає в розгляді конфлікту як явища протяжного у
часу, динамічного процесу прийняття рішень протиборчими сторонами.
Разом з класичними повторюваними іграми останнім часом активно
вивчаються гри на тимчасових шкалах, тобто iгри, в яких час для одного або
всіх гравців влаштовано складніше, ніж просто множина $\mathbb{N}_0$.

У даній дипломній роботі, відштовхуючись від відомих результатів по вивченню
моделі Барро --- Гордона на тимчасових шкалах, запропонована і дослiджена модель
взаємодії профспілки і фірми - монополіста.  Розроблене програмне
забезпечення дозволило шляхом імітаційного моделювання дослiдити властивості
запропонованої моделі та сформулювати певні висновки про способи
формування довгострокової рівноваги.

Структура роботи наступна: в розділі 1 коротко викладаються елементи теорії ігор і
аналізу на тимчасових шкалах. У Розділі 2 розглядається теоретичний базис
обох розглянутих моделей. У розділі 3 викладаються результати комп'ютерних
експериментів з вивченими моделями, далі формулюються висновки. Завершує
текст роботи список використаної літератури і додаток, в якому наведен
прокоментований код програми.
